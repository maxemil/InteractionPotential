\section{Methods}

\colorframe{\ColorMethod}{Methods}
% python + BioPython
% biopython is fast
% lookup of energies stored

\begin{frame}{Implementation}
	\begin{itemize}[<+->]
	    \item Implemented in Python
    	\item PDB package from BioPython \cite{pdb}
    	\item Parallelized with multiprocessing package
    	\item Datastructures from pandas
    	\item Matrix for pairwise contacts from \cite{Zhang1997}
	\end{itemize}
\end{frame}

\begin{frame}{Implementation cont'd}
	\begin{enumerate}[<+->]
		\item Load and pre-process Reference structure
	    \item Filter PDB file
	    \item For every possible contact pair, check connectivity criterion
	    \item sum over all pairwise energies
	\end{enumerate}
\end{frame}

\begin{frame}{Evaluation}
    \begin{itemize}
        \item Free desolvation energy can imply structural stability
        \item Possible measure for Quality for predicted structures
        \item Evaluated on targets of CASP11 competition
    \end{itemize}
\end{frame}

\begin{frame}{RMSD}
	\begin{itemize}
		\item Alignment of prediction and target structure based on superposition of $C\alpha$ atoms (again BioPython)
		\item RMSD as measure of deviation
		
	\end{itemize}
\end{frame}


% evaluated on CASP prediction, beacause quality˜stability
% based on RMSD of CA superimposition (again BioPython)


